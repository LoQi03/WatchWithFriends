\chapter{Összegz\'es}
A "WatchWithFriends" egy innovatív közösségi platform, amely célul tűzte ki a társas videóélmények megkönnyítését és elősegítését. A szakdolgozatomban részletesen elemeztem és ismertettem egy ilyen jellegű applikáció felépítésének folyamatát, beleértve a technológiai infrastruktúra kialakítását és a felhasználói interakciók optimalizálását.
Ezen kívül a dolgozat részletesen kitér a kliens és a szerver közötti kommunikációs mechanizmusokra, valamint a szerver és az adatbázisok közötti adatcsere folyamatára. Kiemelt figyelmet fordítottam a frontend és a backend összetevők működésének bemutatására. A backend részben a repositoryk, servicek és controllerek összehangolt működését és tesztelhetőségét tárgyaltam. A frontend vonatkozásában pedig a komponens alapú fejlesztési módszertant és annak előnyeit mutattam be, melyek kulcsfontosságúak a modern webes alkalmazások hatékony megvalósításához.
Valamint részletesen tárgyaltam az applikáció buildelési és telepítési folyamatát is. A buildelési fázisban különös hangsúlyt fektettem a Docker technológia alkalmazására, amely segítségével az alkalmazás konténerizált környezetben való futtatását és skálázhatóságát könnyedén kezelhetjük. A telepítési (deployolási) szakaszban pedig a Drone CI/CD eszközt használtam, ami lehetővé teszi az automatizált folyamatos integrációt és kiszállítást, biztosítva ezzel az alkalmazás zökkenőmentes és gyors frissítéseit a különböző fejlesztési és üzemeltetési környezetekben.