\chapter{Fejleszt\'esi Lehetős\'egek}
A digitális kor határterületein az online közösségi élmények új dimenziókat nyitnak meg számomra. Ennek megfelelően, merész innovációkat tervezek a jelenlegi videómegosztásalapú interaktív platformomon. A többcsatornás integrációt előtérbe helyezve, célom a diverz videómegosztó szolgáltatások spektrumának beemelése, ezzel erősítve a platformom átfogó elérhetőségét és rugalmasságát.
További fejlesztésként a kommunikációs interakciók szélesítését tervezem. Ennek keretében az írott üzenetek mellé a hang- és videóalapú kapcsolattartási módokat is integrálni kívánom, hogy a felhasználók még teljesebb körű, multidimenzionális közösségi élményben részesüljenek.
A szoba tulajdonosának jogköreinek kiszélesítése szintén fontos prioritásom. Terveim között szerepel a felhasználói interakciók finomhangolására vonatkozó funkciók fejlesztése, így a szobatulajdonosként képes leszek szabályozni a hozzáférést, valamint rendelkezni a videók sorrendjének átrendezésére vonatkozóan.
Az autentikáció és a hitelesítés megerősítése érdekében a regisztrációs folyamatot egy e-mail alapú hitelesítési lépéssel szeretném kibővíteni. Ez nemcsak a felhasználói fiókok biztonságát növeli, hanem hozzájárul az integritásom és megbízhatóságom erősítéséhez is.
A tartalmi sokszínűség további bővítéseként a felhasználók által feltölthető videók funkcionalitásának implementálását is tervbe vettem, amely újabb szintre emeli a felhasználói tartalomgenerálás lehetőségeit.
Végül, de nem utolsósorban, a prémium tartalomszolgáltatókkal, mint a Netflix, Disney+ és társaikkal való integrációt is vizsgálom, hogy a felhasználók számára lehetőség nyíljon különféle streaming szolgáltatások tartalmának közös megtekintésére, egyetlen központosított platformomon belül.
Mindezek a fejlesztések az elkövetkező időszakban kívánják megvalósítani azt a célt, hogy egy teljes körű, összetett és interaktív közösségi teret hozzak létre, ahol a digitális élmények új horizontjai tárulnak fel a felhasználók előtt.