\chapter{Bevezet\'es}\label{chapter:bevezetes}
A digitális korban élünk, ahol a technológia gyorsan fejlődik
és az embereknek egyre több lehetőségük van az online interakcióra.
Azonban a közösségi médiával és az egyéb digitális platformokkal eltöltött
idő gyakran személytelen és egyoldalú.

A "WatchWithFriends" névre keresztelt alkalmazásom arra hivatott,
hogy ezen változtasson. Az alkalmazás lehetőséget nyújt arra,
hogy a felhasználók együtt nézzenek YouTube-videókat, miközben élő chaten
kommunikálnak egymással.
Az alkalmazásban létrehozott "szobák" révén a felhasználók egyszerűen csatlakozhatnak,
és megoszthatják egymással kedvenc videóikat egy közös lejátszási listán keresztül.
Az alkalmazás nem csupán egy online videómegosztó platformot egészít ki,
hanem egy új közösségi élményt is kínál. Ezzel az eszközzel a barátok és családok,
akár távoli földrészeken is, összehozhatók egy közös élmény kapcsán. A "WatchWithFriends" az együtt töltött idő minőségét kívánja javítani, lehetővé téve, hogy az emberek valós időben osszák meg reakcióikat, érzéseiket egy-egy videóval kapcsolatban.

Továbbá a chat funkcióval a felhasználók azonnal megvitathatják a látottakat,
megoszthatják tippeiket vagy akár következő videó választási ötleteiket is.

A szoftver nem csak szórakoztató, de oktatási célokra is felhasználható.
Tanárok és diákok könnyedén nézhetnek együtt oktatási anyagokat, és azonnal megbeszélhetik azok tartalmát.
Ezen kívül a vállalati prezentációknál is hasznos lehet a közös videómegtekintés funkció.

A "WatchWithFriends" egy olyan platform, ami a társasági interakciókat nem csak kiterjeszti,
de egy új dimenzióba is emeli. Az alkalmazás tehát nem csak egy technológiai újítás, hanem egy közösségi eszköz is,
amely összeköti az embereket függetlenül fizikai távolságuktól.